The \verb=Analysis= module reads the ADC signal and puts out information on how to draw two bars (one for each speaker) that reflect the amplitude of said signal. Since we want bars for before and after modulation, we'll use two instances of the same module. 

The required inputs include a left or right selection signal to specify which stereo channel we're about to analyse, two 16 bit signed ADC signals (left and right), a clock signal that's synced with the \verb=VGA_Driver= since the polling rate of the \verb=Snd_Driver= and \verb=VGA_Driver= differ. The incoming signals are low pass filtered before they are scaled down to better fit the output singals.

There are two 8-bit signed output signals (once again, one left, one right). They determine the height of the bar which the \verb=VGA_Driver= should render.

\verb=lrsel= determines which stereo channel should be read and thus which bar height should be written to at any given time.


According to Nyquist–Shannon sampling theorem, the incoming audio input will be low-pass filtered. The audio is low-pass filtered to get a smooth render of the bars. 
The low-pass filtering will ignore frequencies over VAD DÅ??. forsätt FIR.
