\subsection{Vol\_Bal:current\_vol\_bal}

The Volume/Balance module (\verb=Vol_Bal=) acts as the hub for processing incoming digital audio signals, forwarded from WM8731 via the SndDriver module. As such, \verb=Vol_Bal= also keeps internal registers that holds current volume and balance levels as signed 4-bit values (legal values range from -5 to 5 where 0 represents no adjustment). These registers update via the one-hot coded input signal \verb=kb_input= applied by the \verb=Keyboard= module, and the values they hold are used as signals (\verb=i_volume_lvl=, \verb=i_balance_lvl=) for the internal modules that process the \verb=LADC= and \verb=RADC= inputs. Furthermore, the signals are also directed as outputs connected to the VGADriver module so that they can be rendered on the screen.

The main function of the Volume/Balance module is to make requested adjustments to incoming values \verb=LADC= and \verb=RADC=, which represent a measured amplitude respectively at a certain time. They will first be adjusted for volume by a function $A_{new} = A_{old} * \sqrt{2}^n$ where $A$ is the amplitude and $n$ is the signed value in the volume level register. The new values are forwarded for balance adjustment jointly with a ready signal to inform that the \verb=adj_LADC= (or \verb=RADC=) should be read. Same processing is applied in the \verb=Bal_Adj= module to produce the \verb=LDAC= and \verb=RDAC= outputs conveyed to SndDriver and Analysis.

As a result, the user can digitally adjust volume -15/+15 dB and also decrease volume by another 15 dB on a single left/right audio channel.

(Lastly, this design will be simplified by combining volume and balance adjustment volumes)
