Introduction (~ 1/2 page)

The entire Design Spec should be 5-7 pages

\begin{itemize}
\item Now you are the engineers who receive the requirement specification. What you have to do is to propose a solution that fulfills the requirements of the system.
\item Contrary to the requirements specifications, now you have to think about HOW you are going to meet  the requirements.
\item The design specification must describe the entire system. Think about the main blocks that your system will have, the functionality of each block and the interaction between them, i.e., which information the have to send to each other. Explain the functionality of the blocks and their interaction from a signal processing point of view, i.e., how the audio, video, etc. are processed in each block and which infomation is transmitted between blocks. You can provide som equations to show the algorithms that are applied. Note that this is very different from providing the hardware interfaces between the blocks.
\item Later, think about the difficulties that you will find in hardware and the hardware limitations (timing, bandwidth, word length, etc.) and check that your design is viable. Some calculations may be necessary. For instance, if a requirrement says that the system must be able to delay the audio signal one second, you will probably think of using a memory in order to meet the requirement. Then you should make some calculations to check how big the memory must be and if it fits in the FPGA or if you need to use an external memory.
\item The design specification must be described from the system level. Please, avoid details that are not relevant at that level. Also, make sure that the person who reads the document can get a clear idea of the entire system.
\item As a result, the design specification must be a technical proposal that shows that you have analyzed the problem and found the difficulties that you will face, and provides a first approach to the solution. A good approach for writing the design specification is to present a block diagram of the system, provide a high-level description (at signal processing level) about the functionality of each block and how the blocks interact, and show which requirements present challenges and how you will solve them in hardware.
\end{itemize}

\subsection{text här} % All above to be removed

Project (insert name here) is based around audio signal processing. The audio input and output both go through the WM8731 chip on a DE2 board and the hardware settings are controlled from a PS/2 keyboard and displayed on a VGA screen. The hardware settings to be implemented are a volume control and a balance control. In addition, an interface consisting of the input and output power level along with appropriate indicators as stated in the requirement specification.

In addition, the output sound should also be sent to a Class-D amplifier.
