The system follows the requirement and design specification quite closely. However there are still possibilities to further develop it into an enjoyable piece. There's a bug with the peak level indicator.  At certain changes, the peak level indicator will disappear and begin falling from the top of the bar, as if it had just rendered a peak that was as big  as the entity of the bar. The peak level indicator, feature wise, could when updated have a slight pause at the peak, so it can be observed more clearly before beginning to drop.

The bars of the sound amplitude are currently a gradient drawn on the background image, being covered by a black bar, to display the amplitude in a gradient bar. However, since it's drawn in the image of the background, the gradient can't change. A possible improvement to the aesthetic of the bar would be rendering the gradient with every time the bar is rendered. Of course this requires quite a bit of back tracking because it's a vastly differnt way to draw bars. 

For clarity, if anyone would ever watch bars to learn anything, the signal bars could have their axis marked with units, thus one could see roughly at what amplitude the peak level indicators stop, for example.

Improvements to the volume and balance bars could include adding more levels. It would be an easy improvement to make, seeing as it's a change that doesn't require going back on decisions already made. The balance bar would of course require a slight change in the background image because of the pre drawn boxes for the balance levels. The volume bar doesn't have the same limitation.

From the keyboard, we currently read the button being released. This helps us avoid problems with the same button being pressed multiple times, or held down. A flaw this causes is that you have to press the button multiple times to cause the effect of it to happen multiple times. If you for example would like to switch the balance fully to the left channel from being fully to the right, that's ten key presses. With clever PS/2 handling, support could be added for a button being held down in order to  repeat the command of that button. This would especially be useful if more leveles were added to the volume and balance levels. 

As an improvemtn, the keyboard could be given control of a lot of variables in the system. For example, the coefficient of the lowering of the peak level indicator., the low pass filtering coefficient (in Analysis). This would give the user more control of the system.