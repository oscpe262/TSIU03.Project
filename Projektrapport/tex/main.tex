%%% Delete notes once the section is finished according to the notes

%% Main Document
\pagestyle{plain}
\setcounter{page}{1}

%%%
\section{Introduction}\label{cha:intro}
%\hrule
%\em 1/2 page
%
%{\LARGE Har du inte läst README.txt (efter 23/10), gör det omgående!}
%\em
%\hrule

%%%
This is the final report treating the final project in the course
TSIU03 System design. The project was to design and implement a device
used for audio signal processing.

Using a DE2 board an external audio source can be connected via a
3.5 mm port, and accepts input for volume, balance and mute settings
via a PS/2 keyboard the signal is then processed by the application
and an image is rendered on a VGA-monitor displaying four signal level
bars representing the left and right channels both before and after
manipulation. There is also one bar displaying the current volume and
one bar displaying balance setting. There is also a symbol showing if
the system is muted.

This report covers all aspects of the work involved in the
project. It contains a brief system description, it lists the
major challenges and solutions to said challenges and personal
experiences from the group members involved.

%%%
\section{Achievements}\label{cha:achievements}
%%%
%\hrule
%\em
%1/2 page

%Outline the technical challenges that you have solved in your project. This is not the same as the requirements. For instance, it is not a challenge to “use the DE2 board”, it is not a challenge to “use 16 bits for the audio signal”. But it is a challenge to “generate an echo of 0.1 to 3s” or “display numbers on the screen” or “display bars for a signal level indicator on the screen”. Additionally, to have a “smooth movement of the signal level indicator bars on the screen” is an extra achievement apart from just displaying the bars. There are two tables. In the first one (Achievements) write the technical challenges that a user can notice (like the previous ones). In the second table (Design Challenges) write the challenges of the hardware design that cannot be observed from the outside. Like “use the SRAM to store both an image and audio data”, or “implement a low-pass filter with cut-off frequency 500 Hz”, or “transform the value of a sound sample into a coordinate in the screen for the oscilloscope”, or “determine the size of the memory for...”, or “codify the image in a way that uses a small amount of memory”. Note that some difficulties found during the project may not be technical challenges. Some examples are “much time spent on debugging” or “it was difficult to coordinate the group”. Please, keep these difficulties for Subsection 7, where the Project Execution is evaluated.
%\em
%\hrule

%%%
\subsection{User Noticable Achievements}\label{sec:userachievments}
\begin{itemize}
\item Remarkably astonishing graphics.
\item Power level indicators for both the input and output sound signal. 
\item The indicators are moving at a smooth rate with absolutely no flickering.
\item Displaying the current volume and balance adjustments made in a user friendly way.
\item Both the minimum and maximum level of adjustments available to the volume and balance is visible to the user.
\item A mute button is shown if the volume is set to be muted.
\item Output sound is sent to a class-D amplifier.
\item Each bar has a peak level indicator.
\item The peak level indicator falls slowly when the power bar is lower than the peak level indicator.
\end{itemize}


\subsection{Design Challenges}\label{sec:designchallenges}
\begin{itemize}
\item Use the SRAM to store information about a background picture.
\item Implement a low-pass filter for our signal level indicators. 
\item Bar graph rendering (which pixels to write/blank).
\item The logic of adjusting the volume and balance correctly.
\item Detect each key press only once to prevent drastic changes.
\end{itemize}

\section{System Level Description}\label{cha:syslvl}
\begin{figure}[H]
  \centering
  \includegraphics[width=16cm]{overview}
  \caption{G41 Project System Overview}
  \label{fig:overview}
\end{figure}

The project consists of five major modules -- \texttt{Keyboard, Vol\_Bal, Snd\_Driver, Analysis, }and \texttt{VGA\_Driver}. Presented here is an overview of the modules, and detailed functionality is further described in the \citeD.

The system takes, with help of \texttt{Snd\_Driver}, digitally encoded sound from the WM8731 chip. The sound data is passed on to the first instance of \texttt{Analysis} which provides (in order to smoothen the bar movements) low-pass filtered control signals to \texttt{VGA\_Driver:bartender} to assist the rendering of the pre-processing bar graphs done by \texttt{VGA\_Driver} as a whole. The audio data from \texttt{Snd\_Driver} is also passed on to \texttt{Vol\_Bal} which adjusts the audio balance and volume according to register values set by the control signals from \texttt{Keyboard}. The adjusted audio signals are then sent back to \texttt{Snd\_Driver} which sends signals for the class-D amplifier through the GPIO:s. The adjusted audio is also forked off to a second \texttt{Analysis} instance, which assists the post-processing bar graph rendering.

Whereas \texttt{Keyboard} and \texttt{Analysis} already are fairly small and monolithic modules, and \texttt{Snd\_Driver} and \texttt{Vol\_Bal} consists of three submodules each, \texttt{VGA\_Driver} consists of no less than twelve submodules.

\texttt{Keyboard} handles the PS/2 keyboard user input. The module filters break scan codes (F0$_{16}$, XX$_{16}$) bytewise (11 bit/byte) and compares the XX$_{16}$ byte (\emph{iff} directly preceeded by F0$_{16}$) to the control key values. This approach will allow the system to respond to the regular keys (ARROW keys and END) as well as corresponding numeric keypad keys as initial E0$_{16}$ bytes are discarded. Upon a registered valid key release, a 5-bit control signal (\verb=kb_input=) is sent for a single clock cycle to \texttt{Vol\_Bal}.

\texttt{Analysis} reads the \verb=ADC= signals, takes the absolute value (by squaring the value) and low-pass filters these values with a saturation time of approximately 100 ms (4096 samples) (See also \emph{\ref{sec:just.anal} Justification Analysis}). The filtered signals are converted to a graph height (0 to 171 pixels) (see \ref{sec:just.anal}) and passed on to \verb=bar_tender=.

\texttt{Snd\_Driver} have the three submodules \verb=Snd_Driver:{Ctrl,Channel_Mod}= if we count the two instances of \texttt{Channel\_Mod}. These are verbatim copies of \emph{Laboration 4}. The only difference is that the outputs are sent to GPIO pins (and from there via an i2c adapter to the class-D amp) as well as back to the WM8731 codec.

\texttt{Vol\_Bal} consists of the three submodules \texttt{Current\_Vol\_Bal, \{Volume,Balance\}\_Adjustment}. \verb=Current_Vol_Bal= reads \verb=kb_input= and updates registers representing system levels of volume, balance and mute.
\verb=Volume_Adjustment= adjusts volume according to the \texttt{volume\_level} input from \verb=Current_Vol_Bal=, decreasing amplitude by up to -30 dB in 3 dB decrements, and \verb=Balance_Adjustment= adjusts the audio according to system balance level, resulting in a linear scaling of the incoming amplitude. The stronger a channel bias, the higher the signal reduction of the opposing channel. The affected channel loses 1/8 of the amplitude per level of bias.

\texttt{VGA\_Driver} with its submodules handles the rendering of the UI. The module is, in essence, a modified version of the module used in \emph{Laboration 3}. The most significant difference is found in the two new modules \verb=VGA_Driver:Bar_Tender= and \verb=VGA_Driver:Bar_Mixer=. \verb=Bar_Tender= uses the output from \verb=Vol_Bal= and each of the \verb=Analysis= modules along with \verb={h,v}cnt= to calculate which pixel is to be rendered and if it should be rendered from the background image or if to draw it black. The result of the calculations is the control signal \verb=render_bar=. This signal is then passed to \verb=Bar_Mixer=, which is basically a multiplexer, either forwarding the loaded background colours or a blacked out pixel depending on \verb=render_bar= from \verb=Bar_Tender=.

%%%
%\em 1 to 2 pages\\
%Block diagram + description\em
%%%
\section{Justification of the Achievements}\label{cha:justice}
%%%
\hrule
\em 4 to 6 pages

In this subsection, explain how you have solved the challenges described in Subsection 2. It is important that you show how you got from the challenge to a specific solution, and do not miss steps. From the challenge, to the algorithm (mathematical explanation or model), maybe doing some tests in Matlab to take design decisions and validate that the algorithm works. Include the mathematical equations if the algorithm is not obvious. From the algorithm to the hardware implementation, making the numbers of sizes for memories, timing for the signals, design alternatives, decision that you took and why, etc. Include relevant pictures of the circuits or sub-modules of the system to support the explanations. IMPORTANT: You can structure this subsection as a description of the sub-modules of your system, but you have to be aware that what is really important is that you explain clearly how you solved the challenges that you found. \em
\hrule
%\textbf{Skriv i separata filer, inkludera i justification.tex}
%%%
%\input{tex/foo.tex}
\subsection{Justification Keyboard}
In order to detect and register a keypress only once, the \verb=Keyboard= module had to differ signific\-antly from \emph{Laboration 2}. Two viable options were considered: Either, a control for typematic rate effectively reducing the polling rate, else a ``detect-on-release-only''  would have to be implemented. The latter approach was decided upon.

The basic functionality of the module is the same as in the laboration, with the scan codes going through a shift register and being checked against a set of 8-bit values. These values correspond to the control keys plus the break byte F0$_{16}$.

When a start bit reaches the end of the shift register, the shift register is set to reset to 1:s next \texttt{PS2\_CLK} cycle, allowing us to detect each new byte as they are shifted through. This results in the shift register being ``unusable'' for one \texttt{PS2\_CLK} cycle. We circumvent this problem by using a 10 bit (1 bit less than a full sent byte) and ignoring the stop bit.

Next, the current 8-bit value following the start bit is checked. If it equals F0$_{16}$, a register \texttt{BREAKSET} is set in order to enable the next byte to be checked against the control key values. If at the next byte cycle \texttt{BREAKSET} is not set, the byte is ignored. If set, \texttt{BREAKSET} is reset to '0', the byte checked and \verb=kb_input= set to corresponding value.

``Outside'' this \texttt{PS2\_CLK} cycle, the process is controlled by the regular clock. This allows us to again set \verb=kb_input= to idle (all '0') each system clock cycle, which in turn allows us to restrain \verb=kb_input= to be anything but 0 for no more than 1 clock cycle and therefore never registered by \verb=Vol_Bal= more than once per key press.


\subsection{Justification VGA}

One challenge was the image rendering on the VGA-screen. Since a
pre-stored background image is being used, it was decided that the
easiest way to implement the signal level indicators with a gradient
was to create a background image containing filled bars and render
black over the parts that should not be filled rather than having the
application draw the bar itself.

\begin{figure}[H]
        \centering 
        \includegraphics[scale=1.00]{fig/picture_xy.png}
        \caption{Concept UI image}
        \label{fig:picture_xy}
\end{figure}

\verb+figure+ \ref{fig:picture_xy}. is a concept version of the image
used. Constants were declared using the coordinates of the upper right
corner of the bars displaying signal level, both right and left
channel before manipulation, the volume bar and the mute symbol and
the width height and offset to the remaining bars. This way conditions
could be defined for when \verb+hcnt+ and \verb+vcnt+ coordinates were
inside the bars or not.

The input consists of four 8 bit unsigned with values spaning from 0
to 171 that holds information about the signal level bars, a 4 bit
unsigned for volume setting spanning from 0 to 9 and a 5 bit signed
value for balance setting spanning from -8 to 8.

For the signal level bars the application checks that \verb+hcnt+ and
\verb+vcnt+ is in the bar and then compares the \verb+vcount+ value with
(171 -- input)
(which will represent the part of the bar that should be
painted over). And if \verb+hcnt+ and \verb+vcnt+ is in the correct
area the output \verb+render_bar+ is set high and black will be rendered in
the affected pixel.

The volume bar works in a similar way but the input is multiplied with
a constant declaring the width of the boxes, and checking towards
\verb+hcnt+ instead since the bar is positioned horizontally rather
than verticaly.

The balance works as volume with the exception that it's divided in two
parts, one representing the right side being filled and one
representing the left.

There is also support for peak level indicators. They are represented
by 8 bit unsigned internal signals that are assigned the bar value iff
the current bar value is greater than the last value assigned. A
counter counting 50k clock cycles is then decremented before the peak
level amplitude is decremented. The application draws the peak level
line in the same manner as for the bars, but setting the output
\verb+render_peak+ high instead to give the peak level indicator a
different color than the one generated from \verb+render_bar+.

Since the application used in \emph{laboration 2} also rendered a background image
stored in the SRAM only minor ajustments needed to be done. The
sub module \verb+bar_tender+ described above sets \verb+render_bar+ and
\verb+render_peak+ high in the affected pixels and the submodule
\verb+bar_mixer+ works as a multiplexer forwarding color-information
from SRAM when \verb+render_bar+ and \verb+render_peak+ are set
low. And forwards black (for bars) and white (for peak) when the
inputs are set high.

The mute symbol is implemented simply by drawing a black square over
the symbol whenever the \verb+mute+ input is set low.
 


\subsection{Justification Analysis}\label{sec:just.anal}

One challenge that we had through the project were to implement a low-pass filter to get smooth moving bars. During the design part we spent a lot of time thinking of how to do this and calculating math.

The incoming signals are low-pass filtered with a saturation time of approximately 100 ms (4096 sample cycles) as seen in figure \ref{fig:lowpass}, resulting in a measurement of the signal's power. 

\begin{figure}[h]
\centering
\includegraphics[width=16cm]{lowpass}
\caption{The low-pass filter. $k$ is chosen by the approximation $\frac{1}{10}\mathrm{\ s} = 2^k\cdot\frac{1}{48800}\Rightarrow 2^k=4880\approx 2^{12}\Rightarrow k = 12 $}
\label{fig:lowpass}
\end{figure}


\verb=log_Pow= takes the logarithm of the low-pass filtered signal. It provides module outputs proportional to the result, updating only in sync with \verb+vsync+. The \verb+bar+ signals are lined to \verb+VGA_Driver+ as definitive information about how the power bars should render on screen.






\section{User Interface}\label{cha:ui}
%%%
%\hrule
%\em 1 page
%How to control the system + image visualized on the screen
%\em
%\hrule
%%%
The system is controlled by five keys on a PS/2-connected keyboard. Down in the table you can see which key that changes what in the system.


\begin{figure}[h]
\centering
%\caption{PS/2 keys and how the input affects the system.}
\begin{tabular}{|c|c|}
\hline
KEY & Function\\ \hline
U ARROW & Volume Increase\\ \hline
L ARROW & Balance Bias Left\\ \hline
D ARROW &  Volume Decrease\\ \hline
R ARROW &  Balance Bias Right\\ \hline
END		&  Mute Volume\\ \hline
\end{tabular}
\caption{PS/2 keys and how the input affects the system.}
\label{fig:scancodes}
\end{figure}


The volume level has eleven different stages, exclusive mute. When the volume is at max, the whole bar will be red, and when the volume is decreased by one the bar will decrease to show the current volume.

The balance level has seventeen different stages. Eight for left balance, eight for right balance and one stage when the signal is equal. When the balance is equal for left and right the only thing you will notice is the bar and divider. But when you for example press the left arrow, the left side will be filled and the outgoing volume on the right will decrease.

To indicate that mute is enabled you can see a green speaker with a cross to the left side of the picture, and when mute is disabled there will only be a black fiel. Apparent is that the "after" bars will not show anything, due to nothing is being sent out. 

You will be able to see one peak level indicator for each bar, the peak level indicator falls slowly when the power bar is lower than the peak level indicator. With our peak level indicator our user experience is taken to a whole new level, as you will notice when you use our system you will be overwhelmed by the effect.

\begin{figure}[h]
	\centering
        \includegraphics[scale=1]{UI2.png}
       \caption{User interface}
        \label{fig:user interface}
\end{figure}




\section{Improvements}\label{cha:improvements}
%%%
\hrule
\em 1/2 page

In this subsection, discuss what could be improved in the system and which additional features could be added. \em

Eventuellt minska något (mer än 1/2 sida atm)
\hrule
%%%
The system follows the requirement and design specification quite closely. However there are still possibilities to further develop it into an enjoyable piece. There's a bug with the peak level indicator.  At certain changes, the peak level indicator will disappear and begin falling from the top of the bar, as if it had just rendered a peak that was as big as the entity of the bar. The peak level indicator, feature wise, could when updated have a slight pause at the peak, so it can be observed more clearly before beginning to drop.

The bars of the sound amplitude are currently a gradient drawn on the background image, being covered by a black bar, to display the amplitude in a gradient bar. However, since it's drawn in the image of the background, the gradient can't change. A possible improvement to the aesthetic of the bar would be rendering the gradient with every time the bar is rendered. Of course this requires quite a bit of back tracking because it is a vastly different way to draw bars. 

Improvements to the volume and balance bars could include adding more levels. It would be an easy improvement to make, seeing as it's a change that does not require going back on decisions already made. The balance bar would of course require a slight change in the background image because of the pre drawn boxes for the balance levels. The volume bar doesn't have the same limitation.

From the keyboard, we currently read the button being released. This helps us avoid problems with the same button being pressed multiple times, or held down. Therefore, you have to press the button multiple times to cause the effect of it to happen multiple times. If you for example would like to switch the balance fully to the left channel from being fully to the right, that is ten key presses. With clever PS/2 handling, support could be added for a button being held down in order to  repeat the command of that button. This would especially be useful if more levels were added to the volume and balance levels. 

As an improvement, the keyboard could be given control of a lot of variables in the system. For example, the coefficient of the lowering of the peak level indicator, the low pass filtering coefficient (in \texttt{Analysis}). This would give the user more control of the system.


\section{Evaluation of the Project Execution}\label{cha:eval}
%%%
\hrule
\em 1/2 page - komprimera eventuellt 

%In this subsection, discuss how the execution of the project was, if it was according to the original plan, how was the coordination between the members, the time spent of each member in his/her tasks (you have the table below for that), difficulties that you faced, what took more time than expected, why, what took less time than expected, what was easier/more difficult than expected.
\em
\hrule
Time spent for each member, and respective member's tasks:

\begin{figure}[h] % Kunde inte låta bli att pilla. Sorry! ;) /OP
  \centering
  \begin{tabular}{|l|l|c|}
    \hline
    Group Member & Main Tasks & Total Time \\
    \hline
    Niklas Blomqvist & Analysis code & $\sim$65\\
    Philip Johansson & Analysis code & $\sim$70\\
    Matteus Laurent & Vol/Bal code, project coordinating & $\sim$75\\
    Johan Levinsson	& Top TB code &	$\sim$55\\
    Oscar Petersson	& Keyboard code, documentation & $\sim$70\\
    Erik Peyronsson	& VGA code modifications & $\sim$70 \\
    \cline{3-3}
    && 405 h\\
    \hline
  \end{tabular}
\end{figure}

The overarching theme of the project execution would have to be the difficulties faced when one or more members' participation was prevented by illness or similar obstacles, ensuring a less than ideal attendance. Our project group was set back early on because of this very reason, leading to us trailing the schedule by approximately a week. However, the effort invested to catch up was admirable and sufficient, allowing for the eventual completion of the project. Aside from this, the project went mostly according to plan, with a few notable exceptions.

Cooperation and coordination within the group was satisfactory. The project manager was given sufficient authority to lead the distribution and coordination of tasks. A certain level of trust combined with good design of the module overview allowed for more individual workloads, without sacrificing shared communication and system understanding.

All things considered, the design specification was the task that exceeded its set aside time the most. This was due to the document outlining a lot more module details than originally planned, ultimately giving an indepth view of the system. The document was not only expanded upon on recommendation from our project supervisor, but also because of our own natural inquiry when discussing the design overview. Laying the groundwork this way gave voice to design uncertainties early on, and thus helped a lot with subsequent tasks.

Thanks to a thorough design specificiation, the foundation for our first prototype essentially wrote itself. The initial stages of coding were completed swiftly, making up for any time lost (and more) on writing the design document.



%%%
\section{Personal Experiences}\label{cha:personalexp}
%%%
\hrule\em
1/2 to 1 pages - eventuell komprimering 

%In this subsection, include your experiences. What did you liked most in the project? What you have learned? Did you learn from some mistakes/failures? Did you have a good time? Is there something in the course that should be improved? IMPORTANT: The personal experiences are individual and each student in the group has to write a personal paragraph in this subsection. If you want to add more information, or the information is sensitive, please, talk to the Course Responsible.\em
\hrule

%%%
\subsection{Niklas}

With a well written design the conversion from text to VHDL went super smooth and was therefore very enjoyable. One thing I will take as a lesson is the fact that you should consider which different ways of implementing the same problem there is, consider the pros and cons of each way and from there on pick the most suitable one. 
I definitely had a good time working on this project.
\subsection{Philip}
 We spent a lot of time on design. We discussed a lot in the group and calculated on much, the whole design was very well thought out and I an had very good control throughout the system. This led to the implementation going very well, and there not being much trouble in assembling the system. This is something I will take with me in life.

\subsection{Matteus}
All things considered, the project was enlightening with many valuable lessons. Translating the design into our first prototype was quick and done with ease thanks to our design specification - thus emphasising the importance of preparation and good structuring.

Some specific knowledge learned: Modelsim is a powerful tool in resolving bugs and ensuring correct design implementation; word lengths and overflow are the sources of many problems; timing diagrams are very useful.

As for improving the course - we were advised to not use a certain state machine implementation shown in the lectures.
	

	
\subsection{Johan}
The most enjoyable part for me was seeing the shift from design specification to VHDL code. In my head it would not go as easily as it turned out to. If there is anything I wil make sure to learn from, it is going to be writing good design speicifcations. I feel that it should be applicable to other projects.
\subsection{Oscar}
Foremost, the project has significantly increased my comprehension of digital circuit design. It has also shown that (which I am sure all of us already knew) that preparation is indeed key to success. Considering my fetish of documentation, I can not truthfully say that the sparse form of documentation has been a favourite of mine, but considering the given time frame, the intention with the project, and the fact that the course runs in parallel with TDDI02, I have to give it a pass. Over all, it has been a greatly enjoyable course.

\subsection{Erik}
As previous members have noted the design was very well thought
through and the implementation only differed in minor areas. My
biggest challenge was to operate the software, modelsim especially. I
probably should have payed more attention in the labs on how to
operate it cause getting it started and getting the test-bench working
took almost as long as the testing itself wich felt like a waste of time.


%%% Examples for bibliography

%% Bok
% Bryman, Alan (2002),
% \emph{Samhällsvetenskapliga metoder}.
% Malmö: Liber Ekonomi

%% Antologi
% Florin, Cristina & Johansson, Ulla (1996),
% Tre kulturer – tre historier: Disciplinering i läroverk, flickskolor och folkskolor under 1800-talets senare hälft.
% I S.G. Nordström (red.), \emph{Utbildningshistoria}. Årsböcker i svensk undervisningshistoria, vol. 182.
% Uppsala: Reprocentralen HSC

%% Tidsskrift
% Eriksson, Lisbeth (1999),
% Deltagarna i Kvinnonavet.
% I Susanne Köpsén (red.) \emph{ Motdrag i förort -– ett folkbildningsprojekt i ett mångkulturellt bostadsområde. }
% Linköping: Linköpings universitet. Vuxenutbildarcentrums skriftserie nr. 12/99

%% Fler exempel
% Se Svenska Skrivregler 5.2
% Referenser i engelska texter skiljer sig egentligen en aning, men det bortser vi ifrån i det här fallet.

%%% Bibliography
\renewcommand*{\refname}{References to the Project Files}
\begin{thebibliography}{99}\label{cha:refs}

\bibitem{design}
  \href{https://github.com/oscpe262/TSIU03.Project/blob/master/Designspec/designspec.pdf}{
    Group 41 - Design Specification,
    \emph{Linköping},
    2015-10-19.
  }

\bibitem{reqspec}
  \href{https://github.com/oscpe262/TSIU03.Project/blob/master/Kravspec/Kravspecifikation.pdf}{
    Group 41 - Requirement Specification,
    \emph{Linköping},
    2015-09-22.
  }
  
\bibitem{pres}
  \href{https://github.com/oscpe262/TSIU03.Project/blob/master/Presentationer/firstpres.pdf}{
    Group 41 - First Presentation,
    \emph{Linköping},
    2015-10-15.
  }
  
\bibitem{plan}
  \href{https://github.com/oscpe262/TSIU03.Project/blob/master/Projektplan/Project.plan.pdf}{
    Group 41 - Project Plan,
    \emph{Linköping}
    2015-09-24.
  }
  
%\bibitem{vhdldummy}
%  \href{https://github.com/oscpe262/TSIU03.Project/blob/master/<directory>/<filename>}{
%    ModuleName,
%    \texttt{filename.vhd}
%  }
  
\end{thebibliography}
\let\thefootnote\relax\footnote{All documents available at \url{https://github.com/oscpe262/TSIU03.Project}. The references in the PDF version of this document are hyperlinks to the document listed. The PDF can be downloaded at: \\\url{https://github.com/oscpe262/TSIU03.Project/blob/master/Projektrapport/report.pdf}.}


